	\chapter{Conclusões e trabalhos futuros}
	
	Durante o desenvolvimento deste trabalho de pesquisa ficou evidente a ausência de dispositivos que pudessem ser utilizados para o estudo, a análise e a pesquisa dos sistemas baseados no paradigma evolutivo. Também verificou-se a necessidade de uma metodologia de pesquisa.
	
	Essa conclusão deveu-se a partir de um estudo sistemático em cinco etapas: a Etapa 1 contribuiu com uma visão mais realista do contexto global e identificou o problema para ser tratado no projeto de pesquisa; a Etapa 2 permitiu identificação dos sistemas, das ferramentas e das metodologias que estavam, naquele momento realizando sistemas evolutivos; na Etapa 3 chegou-se à visão da influência dos sistemas evolutivos, e suas propriedades, para o problema da customização em massa, e tais propriedades puderam ser estudadas e entendidas; na Etapa 4, os estudos realizados até então foram transformados em questões de pesquisa e hipóteses, as quais, por sua vez, foram transformadas em objetivos; na Etapa 5, os objetivos foram refinados em requisitos para EPS e notou-se a ausência de uma metodologia para desenvolvimento de sistemas evolutivos, o que levou ao desenvolvimento do MeDSE e, na sua aplicação, ao desenvolvimento do SIAPE, objeto principal deste trabalho.
	
	A primeira questão de pesquisa investigou a característica de sistemas ágeis, que no curto prazo, pudesse evidenciar um sistema evolutivo de produção e a hipótese formulada foi que a adaptabilidade conseguida através da reconfiguração de parâmetros do sistema conseguiria evidenciar um sistema evolutivo de produção.
	
	Na análise da experimentação, quando o SIAPE finalizou o produto A e iniciou o produto B, todas as alterações para realizar a operação de troca de produto aconteceu na parte virtual do sistema, isto é, o agente Order instanciou o agente Anagrama adequado, e a operação foi realizada com a devida agilidade e flexibilidade de um sistema evolutivo.
	
	Na segunda questão de pesquisa foi investigada a característica de sistemas que, no longo prazo, identificasse a realização de um sistema evolutivo. A hipótese formulada foi que a evolutividade seria, nos sistemas evolutivos, realizada por meio da inclusão ou exclusão de módulos no sistema sem comprometer a eficiência e o funcionamento do sistema.
		
	Na análise do experimentação, quando o SIAPE finalizou o produto B e teve inicio o produto C, a letra E não estava disponível no hardware do SIAPE. Para tratar esse problema foi realizado o conceito de \textit{plug-and-produce}, isto é, foi inserido  o módulo da letra E no hardware do SIAPE, a interface com o operador atualizada e a letra E disponibilizada para o sistema. O agente Order, com essa informação, instanciou o agente Anagrama adequado e este, em conjunto com o agente Acesso Hardware realizaram o produto C (a palavra UEA). Percebe-se aqui que, com a repetição deste \textit{skill}, pode-se trocar todas as letras do sistema, os circuitos elétricos, formas dos módulos, até mesmo evoluir todo o sistema para um novo sistema detentor de uma melhor performance ou melhor tecnologia, admitindo somente que a comunicação entre os módulos não seja quebrada.
	
	%A terceira questão de pesquisa investigou o motivo do conceito \textit{plug-and-produce} ser possível apenas em sistemas modulares. A hipótese formulada foi que com módulos independentes uns dos outros pode-se implementar softwares baseados em sistemas multi-agentes para controlar, separadamente cada um desses módulos, possibilitando assim a realização do conceito.
	
	Outro importante conceito evidenciado com o SIAPE foi o conceito da auto-organização, o que se deu em dois momentos.
	
	O primeiro momento foi evidenciado quando o SIAPE realizou os produtos A e B, ou seja, realizou uma autoconfiguração sem qualquer interferência externa. Importante o leitor visualizar as negociações realizadas no mundo virtual dos agentes e as decisões sendo tomadas para que o agente Acesso Hardware realizasse no mundo real, em tempo real as operações necessárias para atingir as metas do sistema.
	
	O outro momento foi quando os agentes negociam para realizar o produto C (UEA) pois o recurso E foi inserido em tempo de produção e houve a necessidade de negociação entre agentes que já existiam para que o sistema se auto-organizasse e continuasse a produção do plano de produção.
	
	Por sua vez a auto-organização remete à emergência do SIAPE pois tanto no caso da produção dos produtos A e B quanto na produção do produto C, o sistema exibiu emergência, pois emergentes coerentes surgiram no nível macro (a produção dos produtos A, B e C)  das interações dinâmicas ocorridas no nível micro (as negociações entre agentes). Deste ponto de vista micro, não há informações suficientes, em nenhum dos agentes, para se montar completamente um produto, e a interação entre os agentes parece aleatória; entretanto esta interação consegue produzir emergentes coerentes no nível macro.
	
	Observar que um sistema em que o agente AcHw detenha o conhecimento de como se gera um produto determinado é voltar ao paradigma tradicional. Se o software puder ser atualizado, teríamos um sistema flexível (\iTresZero). Entretanto, da maneira como determinado pelo paradigma, os agentes foram implementados de tal forma que, nenhum deles tendo a informação completa, mas as suas interações executam o plano de produção, tem capacidade de auto-organização, em especial o \textit{plug-and-produce}, o que corresponde a um sistema ágil (\iQuatroZero).
	   
	%Na análise da experimentação, existe a possibilidade de cinco módulos de letras (A,F,M,T e U) que são atuadores implementados com motores DC e sensores magnéticos que captam a presença dos paletes e permitem que o agente Acesso Hardware acione ou não os atuadores para realizarem as operações solicitadas. Cada \textit{skill} realizado, seja para mover a esteira ou acionar um módulo, são trechos de códigos na linguagem Java que implementam agentes inteligentes que negociam entre si para decidir qual o módulo e qual a posição será carimbada. Caso as letras não fossem modulares, isto é, um hardware completo e independente para cada letra, que possibilita a instância exata da letra e da posição a ser carimbada no palete, o conceito \textit{plug-and-produce} estaria comprometido, pois não haveria a possibilidade de inclusão ou exclusão de módulos no sistema em tempo de processamento. Assim a plataforma Jade implementando Sistemas multi-agentes consegue realizar o conceito da plugabilidade, e por conseguinte a capacidade de se adaptar e de evoluir comum aos sistemas EPS. \par
		
	Com a análise comparativa realizada, ficou evidente que o SIAPE teve como base o Produto UFAM, que foi desenvolvido para demonstrar uma planta fabril automatizada simplificada, comum ao parque industrial brasileiro. %Com a análise relativa ao IADE ficou evidente que as características básicas do paradigma EPS estão presentes no SIAPE, mas devido às características do hardware desenvolvido pode ser expandido e incluídas outras propriedades para que o SIAPE seja reconhecido como um protótipo do paradigma EPS.
	
	Com esse sentimento de melhoria do SIAPE propõe-se como desafio para trabalhos futuros, primeiramente, a realização do SIAPE em escala normal para que possíveis melhorias sejam implementadas no hardware atual, incluindo entre outras características, a detecção de falhas. Num segundo projeto, a realização de uma planta fabril aplicada a um produto do parque industrial brasileiro para a realização dos conceitos e operações de sistemas evolutivos, fato que será acompanhado de outros estudos e pesquisas que evidenciarão a eficácia do paradigma EPS na parte real do sistema, e iniciando uma nova fase nas pesquisas de sistemas evolutivos, dando continuidade à recomendação de Cavalcante na criação de um grupo de pesquisa no Brasil em torno do paradigma evolutivo que alavanque as pesquisas e inovações na Academia Brasileira em torno do tema.
	
	% Espera-se que a realização de trabalhos futuros e o fomento de pesquisas, no modo aqui proposto, facilite a adequação de sistemas e produtos às recomendações da Plataforma da Indústria 4.0 e as empresas brasileiras se tornem mais aderentes aos ambientes 4.0 e consigam tratar o problema customização de massa com maior robustez e eficácia.    
		
	